%In \S\S \ref{sec:reqt_philosophy},\ref{sec:wl_gal-clusters},\ref{sec:gc}, we
%describe our work plan and explain how we will iteratively flow down science objectives to the
%measurements to be conducted, develop observational strategies,
%simulate synthetic astronomical `truth' data and the observational
%data output (including calibration), develop a methodology for
%validating dark energy constraints, and define scientific performance
%requirements and a complete plan for the science investigation.
%Our work plan maps the six SIT tasks into the deliverables below. For each deliverable, we identify in parentheses the required tasks (numbered \textit{T 1-6} as in \S 3.1 of the WFIRST SIT call). We will also associate explicitly the sections of our proposal to the deliverables.

\begin{summary}
We describe in this section the new deliverable we added to the list we initially proposed.
\end{summary}

\subsection*{(ND1) Joint DC2 LSST-WFIRST pixel-level simulations and analysis}
%=============================================================================
\label{sec:joint_simulations}

\paragraph*{Deliverable:} The goal of this deliverable is to produce joint DC2
LSST-WFIRST pixel-level simulations and analysis. DC2 corresponds to the second
LSST DESC data challenge (details can be found here
\href{https://github.com/LSSTDESC/DC2-production/tree/master/Documents/DC2_Plan}{[Link]}).
We will produce a WFIRST imaging simulation matched to DC2, utilizing the
existing image simulation framework built by our SIT. The DESC CosmoDC2 catalog (and
if necessary instance catalogs) will serve as inputs to the WFIRST image
simulation to create matched scenes to the DESC DC2 image simulations. The
output of the WFIRST ImSim, along with the existing DC2 pixel-level simulation
data, will serve as a basis to study joint pixel-level analysis techniques for
measuring shapes and photo-$z$s, in particular methods of deblending. Results
from these studies can also inform improvements to DESC DC input catalogs to
improve realism in the near-IR. This project will lead to the production of
(part of) the following software or dataset deliverable(s): DC2/DC3 image
simulation for weak-lensing and photo-$z$ and supplementary simulation data for
testing joint pixel-level processing with WFIRST. DC2 simulation characteristics 
covers 5,000 sq.~deg, while simulated imaging will be produced for about 300 sq.~deg.
Galaxies are sampled from \texttt{Galacticus} output. Magnitude information is provided in
the LSST $ugrizY$ and WFIRST filters. Both absolute magnitudes in the galaxy
rest-frame and apparent magnitudes in the observer frame are provided for the
SDSS and LSST filters. Magnitudes extincted by host galaxy dust, including WFIRST
magnitudes, and emission-line strengths are computed in post-processing. Many
galaxy properties such as SFR of the disk and bulge, metal mass, black hole mass
and accretion, etc. are available too.

Our SIT decided to create this new deliverable in addition to generating WFIRST HLS
imaging simulations only as it guarantees a broader user community for these
simulations. It is the best way to demonstrate the power of the WFIRST imaging
survey. It is also an important first collaborative effort and a foretaste of
what will be done when actual LSST and WFIRST data are present. It also allows
key LSST and WFIRST experts to jointly analyze these simulations, which will
undoubtedly enable new studies.

\paragraph*{Delivery Date:} CY19, CY20

\paragraph*{Status: On schedule.} Over the last few months we have established that the properties of the galaxies in the current DC2 are good enough when observed through the WFIRST bands. This project was made an official DESC project. We are now in the process of generating a few 5 sq. deg. patches of the sky which would have both LSST and WFIRST imaging. Once this is validated, we will generate 300 sq. deg. of WFIRST imaging to match existing LSST image simulation plans. We will study redshift and shape recovery both individually in each simulated survey and jointly. The outcome of this joint DESC-WFIRST project can then be factored (among other considerations) into the definition of LSST DESC DC3 and future joint simulation efforts. 

\Oli{Troxel, Rachel M. Please correct or add material if need be.}
